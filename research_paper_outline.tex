\documentclass[12pt]{article}
\usepackage{geometry}
\geometry{a4paper, margin=1in}
\usepackage{hyperref}
\usepackage{graphicx}
\usepackage{listings}
\usepackage{amsmath}
\usepackage{natbib}
\usepackage{color}
\hypersetup{colorlinks=true, urlcolor=blue, linkcolor=blue, citecolor=blue}

\title{A Reproducible, Configurable Data Entry and Backup System:\\
Design, Implementation, and Applications Using Literate Programming}
\author{Roy J Wheelock}
\date{\today}

\begin{document}

\maketitle

\begin{abstract}
Summarize the need for systematic data entry and reproducible backup systems. Highlight the design of a configurable, extensible system using rsync, shell scripts, Git, and literate programming, and describe deliverables and application areas.
\end{abstract}

\section{Introduction}
Discuss the challenges of manual data entry, risk of data loss, and the importance of reproducibility. Introduce literate programming and state the goals of this project.

\section{Related Work}
Review backup tools (rsync, restic, Borg), scientific workflow systems (Snakemake, Airflow), and literate programming tools (noweb, org-mode, Jupyter).

\section{System Architecture}
Describe the directory structure and system layout. Include a diagram. Explain each component: configuration, setup, backup, recovery, audit logging, and testing.

\section{Implementation Details}
Provide technical details of shell scripts, rsync usage, JSON audit structure, testing with pytest, and CI/CD considerations.

\section{Literate Programming Process}
Describe how the system code and documentation are combined. Include examples.

\section{Use Case Examples}
Present real and hypothetical use cases: blood pressure logging, garden health monitoring, and laboratory notebooks.

\section{Results and Evaluation}
Evaluate system reliability, ease of setup, reproducibility, and testing outcomes.

\section{Discussion}
Discuss strengths, limitations, scalability, and future improvements.

\section{Conclusion}
Summarize project contributions, impact, and future directions.

\bibliographystyle{plain}
\bibliography{references}

\appendix
\section{Appendix A: Example Configuration File}
\lstinputlisting[language=bash]{config.sh}

\section{Appendix B: Audit Log Example}
\lstinputlisting[language=json]{example_audit.json}

\end{document}
